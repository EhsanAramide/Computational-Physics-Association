% See `LICENCE' file in root directory for more information
% Only use XeLaTeX compiler 
\documentclass[a4paper]{article}

% Packages
% \usepackage{showframe}
\usepackage{hyperref}
\usepackage{array}
\usepackage{xepersian}

% `xepersian'
\settextfont[Scale=1]{XB Zar}
\setlatintextfont[Scale=1]{Junicode}
\rightfootnoterule

% Document class `article'
\title{حلقه‌ی فیزیک محاسباتی\RTLthanks{قدرت گرفته از \XePersian}}
\author{احسان آرمیده}
\date{مهر ۱۴۰۳ (ترم ۴۰۳۱)}

\linespread{1.5}

\begin{document}
\maketitle

\prq{}یک دانشمند نباید به نتایج شبیه‌سازی‌اش باور داشته باشد، زمانی که محاسبات پشت آن را درک نکرده است.\plq{}
{\small \itshape رابین لانداو}\RTLfootnote{Landau H. Rubin}
\vspace{10pt}

با توجه به اهمیت روزافزون فناوری‌های کامپیوتری در پیش‌برد علوم بنیادی همچون ریاضیات و فیزیک، 
این مسئولیت احساس شد تا برای اشتراک‌گذاری و یادگیری این موضوع جالب و مدرن حلقه‌ای تشکیل بشود.
در این حلقه برای یادگیری بهتر و بیشتر، از مطالب ساده‌تر مکانیک کلاسیک شروع میشود. سپس در ترم‌های
آینده به بررسی مباحث پیچیده‌تر، هم از نظر فیزیکی و هم از نظر کامپیوتری، پرداخته میشود.

در این ترم، به بررسی مطالب زیر میپردازیم:

\begin{center}
    \begin{tabular}{ | m{0.7cm} | m{2cm} | m{6cm} | }
        \hline
        ردیف & مبحث & جزییات \\ \hline
        ۱ & مروری بر پایتون & مروری بر Syntax زبان پایتون و محیط‌های توسعه‌ی مربوطه \\ \hline
        ۲ & پکیج‌های محاسبات علمی & پکیج‌های ،NumPy ،SciPy MatPlotLib \\ \hline
        ۳ & انتگرال و مشتق گیری عددی & بررسی روش‌های مختلف انتگرال و مشتق گیری عددی اعم از اویلر، سیمپسون و \dots \\ \hline
        ۴ & حل عددی معادلات دیفرانسیل & حل معادلات دیفرانسیل معمولی یا ODE به صورت عددی \\ \hline
        ۵ & شبیه‌سازی & شبیه‌سازی‌های متنوع و hands-on از سیستم‌های فیزیکی مختلف \\
        \hline
    \end{tabular}
\end{center}

\end{document}


